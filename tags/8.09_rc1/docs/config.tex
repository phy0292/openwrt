\subsubsection{Structure of the configuration files}

The config files are divided into sections and options/values.

Every section has a type, but does not necessarily have a name.
Every option has a name and a value and is assigned to the section
it was written under.

Syntax:

\begin{Verbatim}
config      <type> ["<name>"]      # Section
    option  <name> "<value>"       # Option
\end{Verbatim}

Every parameter needs to be a single string and is formatted exactly
like a parameter for a shell function. The same rules for Quoting and
special characters also apply, as it is parsed by the shell.

\subsubsection{Parsing configuration files in custom scripts}

To be able to load configuration files, you need to include the common
functions with:

\begin{Verbatim}
. /etc/functions.sh
\end{Verbatim}

Then you can use \texttt{config\_load \textit{<name>}} to load config files. The function
first checks for \textit{<name>} as absolute filename and falls back to loading
it from \texttt{/etc/config} (which is the most common way of using it).

If you want to use special callbacks for sections and/or options, you
need to define the following shell functions before running \texttt{config\_load}
(after including \texttt{/etc/functions.sh}):

\begin{Verbatim}
config_cb() {
    local type="$1"
    local name="$2"
    # commands to be run for every section
}

option_cb() {
    # commands to be run for every option
}
\end{Verbatim}

You can also alter \texttt{option\_cb} from \texttt{config\_cb} based on the section type.
This allows you to process every single config section based on its type
individually.

\texttt{config\_cb} is run every time a new section starts (before options are being
processed). You can access the last section through the \texttt{CONFIG\_SECTION}
variable. Also an extra call to \texttt{config\_cb} (without a new section) is generated
after \texttt{config\_load} is done.
That allows you to process sections both before and after all options were
processed.

Another way of iterating on config sections is using the \texttt{config\_foreach} command.

Syntax:
\begin{Verbatim}
config_foreach <function name> [<sectiontype>] [<arguments...>]
\end{Verbatim}

This command will run the supplied function for every single config section in the currently
loaded config. The section name will be passed to the function as argument 1.
If the section type is added to the command line, the function will only be called for
sections of the given type.


You can access already processed options with the \texttt{config\_get} command
Syntax:

\begin{Verbatim}
# print the value of the option
config_get <section> <option>

# store the value inside the variable
config_get <variable> <section> <option>
\end{Verbatim}

In busybox ash the three-option \texttt{config\_get} is faster, because it does not
result in an extra fork, so it is the preferred way.

Additionally you can also modify or add options to sections by using the
\texttt{config\_set} command.

Syntax:

\begin{Verbatim}
config_set <section> <option> <value>
\end{Verbatim}

If a config section is unnamed, an automatically generated name will
be assigned internally, e.g. \texttt{cfg1}, \texttt{cfg2}, ...

While it is possible, using unnamed sections through these autogenerated names is
strongly discouraged. Use callbacks or \texttt{config\_foreach} instead.


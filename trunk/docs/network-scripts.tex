\subsubsection{Using the network scripts}

To be able to access the network functions, you need to include
the necessary shell scripts by running:

\begin{Verbatim}
. /etc/functions.sh      # common functions
include /lib/network     # include /lib/network/*.sh
scan_interfaces          # read and parse the network config
\end{Verbatim}

Some protocols, such as PPP might change the configured interface names
at run time (e.g. \texttt{eth0} => \texttt{ppp0} for PPPoE). That's why you have to run
\texttt{scan\_interfaces} instead of reading the values from the config directly.
After running \texttt{scan\_interfaces}, the \texttt{'ifname'} option will always contain
the effective interface name (which is used for IP traffic) and if the
physical device name differs from it, it will be stored in the \texttt{'device'}
option.
That means that running \texttt{config\_get lan ifname}
after \texttt{scan\_interfaces} might not return the same result as running it before.

After running \texttt{scan\_interfaces}, the following functions are available:

\begin{itemize}
    \item{\texttt{find\_config \textit{interface}}} \\
        looks for a network configuration that includes
        the specified network interface.

    \item{\texttt{setup\_interface \textit{interface [config] [protocol]}}} \\
      will set up the specified interface, optionally overriding the network configuration
      name or the protocol that it uses.
\end{itemize}

\subsubsection{Writing protocol handlers}

You can add custom protocol handlers (e.g: PPPoE, PPPoA, ATM, PPTP ...)
by adding shell scripts to \texttt{/lib/network}. They provide the following
two shell functions:

\begin{Verbatim}
scan_<protocolname>() {
    local config="$1"
    # change the interface names if necessary
}

setup_interface_<protocolname>() {
    local interface="$1"
    local config="$2"
    # set up the interface
}
\end{Verbatim}

\texttt{scan\_\textit{protocolname}} is optional and only necessary if your protocol
uses a custom device, e.g. a tunnel or a PPP device.


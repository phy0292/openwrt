The WiFi settings are configured in the file \texttt{/etc/config/wireless}
(currently supported on Broadcom and Atheros). When booting the router for the first time
it should detect your card and create a sample configuration that looks like this:

\paragraph{Sample Broadcom wireless config:}

\begin{Verbatim}
config wifi-device      "wl0"
    option type         "broadcom"
    option channel      "5"

config wifi-iface
    option device       "wl0"
    option mode         "ap"
    option ssid         "OpenWrt"
    option hidden       "0"
    option encryption   "none"
\end{Verbatim}

\paragraph{Sample Atheros wireless config:}

\begin{Verbatim}
config wifi-device      "wifi0"
    option type         "atheros"
    option channel      "5"
	option mode  		"11g"

config wifi-iface
    option device       "wifi0"
    option mode         "ap"
    option ssid         "OpenWrt"
    option hidden       "0"
    option encryption   "none"
\end{Verbatim}

There are two types of config sections in this file. The '\texttt{wifi-device}' refers to
the physical wifi interface and '\texttt{wifi-iface}' configures a virtual interface on top
of that (if supported by the driver).

\paragraph{Options for the \texttt{wifi-device}:}

\begin{itemize}
    \item \texttt{type} \\
        The driver to use for this interface.
	
	\item \texttt{mode} \\
		The frequency band (\texttt{b}, \texttt{g}, \texttt{bg}, \texttt{a})

    \item \texttt{country} \\
        The country code used to determine the regulatory settings.

    \item \texttt{channel} \\
        The wifi channel (e.g. 1-14, depending on your country setting).

    \item \texttt{maxassoc} \\
        Maximum number of associated clients

\end{itemize}

\paragraph{Options for the \texttt{wifi-iface}:}

\begin{itemize}
    \item \texttt{mode} \\
        Operating mode:

        \begin{itemize}
            \item \texttt{ap} \\
                Access point mode

            \item \texttt{sta} \\
                Client mode

            \item \texttt{adhoc} \\
                Ad-Hoc mode

            \item \texttt{wds} \\
                WDS point-to-point link

        \end{itemize}

    \item \texttt{network} \\
        Selects the interface section from \texttt{/etc/config/network} to be
        used with this interface

    \item \texttt{encryption} \\
        Encryption setting. Accepts the following values:

        \begin{itemize}
            \item \texttt{psk}, \texttt{psk2} \\
                WPA(2) Pre-shared Key

            \item \texttt{wpa}, \texttt{wpa2} \\
                WPA(2) RADIUS

        \end{itemize}

    \item \texttt{key} (wpa and psk) \\
        Either the WPA key (PSK mode) or the RADIUS shared secret (WPA RADIUS mode)

    \item \texttt{server} (wpa) \\
        The RADIUS server address

    \item \texttt{port} (wpa) \\
        The RADIUS server port

\end{itemize}

\paragraph{Limitations:}

Only the following mode combinations are supported:

\begin{itemize}
    \item \textbf{Broadcom}: \\
        \begin{itemize}
            \item 1x \texttt{sta}, 0-3x \texttt{ap}
            \item 1-4x \texttt{ap}
            \item 1x \texttt{adhoc}
        \end{itemize}

        WDS links can only be used in pure AP mode and can't use WEP (except when sharing the
        settings with the master interface, which is done automatically).

    \item \textbf{Atheros}: \\
        \begin{itemize}
            \item 1x \texttt{sta}, 0-4x \texttt{ap}
            \item 1-4x \texttt{ap}
            \item 1x \texttt{adhoc}
        \end{itemize}

\end{itemize}


